%%%%%%%%%%%%%%%%%%%%%%%%%%%%%%%%%%%%%%%%%
% Thin Sectioned Essay
% LaTeX Template
% Version 1.0 (3/8/13)
%
% This template has been downloaded from:
% http://www.LaTeXTemplates.com
%
% Original Author:
% Nicolas Diaz (nsdiaz@uc.cl) with extensive modifications by:
% Vel (vel@latextemplates.com)
%
% License:
% CC BY-NC-SA 3.0 (http://creativecommons.org/licenses/by-nc-sa/3.0/)
%
%%%%%%%%%%%%%%%%%%%%%%%%%%%%%%%%%%%%%%%%%

%----------------------------------------------------------------------------------------
%	PACKAGES AND OTHER DOCUMENT CONFIGURATIONS
%----------------------------------------------------------------------------------------

\documentclass[a4paper, 11pt]{article} % Font size (can be 10pt, 11pt or 12pt) and paper size (remove a4paper for US letter paper)

\usepackage[protrusion=true,expansion=true]{microtype} % Better typography
\usepackage{graphicx} % Required for including pictures
\usepackage{graphicx}
%\usepackage[font=small]{subfig}
\graphicspath{{images/}}
\DeclareGraphicsExtensions{.pdf,.png,.jpg}

\usepackage{wrapfig} % Allows in-line images

\usepackage{mathpazo} % Use the Palatino font
\usepackage[T1]{fontenc} % Required for accented characters
\linespread{1.05} % Change line spacing here, Palatino benefits from a slight increase by default

\makeatletter
\renewcommand\@biblabel[1]{\textbf{#1.}} % Change the square brackets for each bibliography item from '[1]' to '1.'
\renewcommand{\@listI}{\itemsep=0pt} % Reduce the space between items in the itemize and enumerate environments and the bibliography

\renewcommand{\maketitle}{ % Customize the title - do not edit title and author name here, see the TITLE block below
\begin{flushright} % Right align
{\LARGE\@title} % Increase the font size of the title

\vspace{50pt} % Some vertical space between the title and author name

{\large\@author} % Author name
\\\@date % Date

\vspace{40pt} % Some vertical space between the author block and abstract
\end{flushright}
}

%----------------------------------------------------------------------------------------
%	TITLE
%----------------------------------------------------------------------------------------

\title{\textbf{Electrical Generator Technologies for Offshore Wind Turbines}} % Subtitle

\author{\textsc{Ozan Keysan} % Author
\\{\textit{Institute for Energy  Systems\\ University of Edinburgh}}} % Institution

\date{\today} % Date

%----------------------------------------------------------------------------------------

\begin{document}

\maketitle % Print the title section

%----------------------------------------------------------------------------------------
%	ABSTRACT AND KEYWORDS
%----------------------------------------------------------------------------------------

%\renewcommand{\abstractname}{Summary} % Uncomment to change the name of the abstract to something else

\begin{abstract}
Morbi tempor congue porta. Proin semper, leo vitae faucibus dictum, metus mauris lacinia lorem, ac congue leo felis eu turpis. Sed nec nunc pellentesque, gravida eros at, porttitor ipsum. Praesent consequat urna a lacus lobortis ultrices eget ac metus. In tempus hendrerit rhoncus. Mauris dignissim turpis id sollicitudin lacinia. Praesent libero tellus, fringilla nec ullamcorper at, ultrices id nulla. Phasellus placerat a tellus a malesuada.
\end{abstract}

\hspace*{3,6mm}\textit{Keywords:} lorem , ipsum , dolor , sit amet , lectus % Keywords

\vspace{30pt} % Some vertical space between the abstract and first section

%----------------------------------------------------------------------------------------
%	ESSAY BODY
%----------------------------------------------------------------------------------------

\section*{Introduction}

In this chapter different electric generator options for offshore wind turbines will be reviewed. The mechanical energy captured by the turbine blades are converted to electrical energy and is fed into the grid with a series of components such as; gearbox, generator, power electronics, transformer and transmission. Generator is the most important component in the power take-off system and there are many different types of generator types that can be used in turbines depending on the size, operating conditions, and speed range. Some common drive train types are tabulated in Table \ref{generator_manufacturers}.

In this chapter, conventional generator technologies will be reviewed as well as some novel generator designs.


%GE machine specs
\begin{table}[t]
  \centering
  \begin{tabular}{clrrlc}
	  
	  Drive Train & Generator & Power & Rotor & Rated & Manufacturer \\
	  \hline
		& DFIG & 2 MW & 90 m & 20.7 rpm & DeWind \\
		& DFIG & 2 MW & 90 m & 19 rpm & Gamesa \\
	  	& WRIG & 2 MW & 88 m &  & Suzlon \\
	  	& PMSG & 2 MW & 88 m & 16.5 rpm & General Electric \\
	 Multiple 	& DFIG & 2.5 MW & 90 m & 14.9 rpm & Nordex \\
	Stage  	& DFIG & 3 MW & 100 m & 14.3 rpm & Ecotecnia \\
  	Gearbox & DFIG & 3.6 MW & 104 m & 15.3 rpm & General Electric \\
	  	& DFIG & 4.5 MW & 120 m & 14.9 rpm & Vestas \\
	  	& DFIG & 6 MW & 126 m & 12.1 rpm & RePower \\
	  	\hline
	Single--Stage  	& PMSG & 3 MW & 90 m & 18 rpm & Winwind \\
	Gearbox  	& PMSG & 5 MW & 116 m & 14.8 rpm & Multibrid \\
	  	\hline
	 Hydraulic 	& EESG & 2 MW & 90 m & 20.7 rpm & DeWind \\
	 Transmission 	& EESG & 2.4 MW &  & 10 rpm & Mitsubishi \\
	  	\hline
	  	& PMSG & 1.5 MW & 77 m & 17.3 rpm & Vensys \\
	  	& EESG & 1.6 MW & 78 m &  & MTOI \\
	  	& PMSG & 2 MW &  &  & Mitsubishi \\
	Direct-Drive  	& PMSG & 2 MW & 71 m & 23 rpm & Zephyros \\
	  	& PMSG & 3 MW & 101 m & 14.4 rpm & LeitWind \\
	  	& EESG & 4.5 MW & 114 m & 13 rpm & Enercon \\
	  	& EESG & 7.5 MW & 127 m & 11.7 rpm & Enercon \\
	  \hline

  \end{tabular}
  \caption{Drive train types of some wind turbines \cite{wind_energy_facts,upwind2011}.}
  \label{generator_manufacturers}
\end{table}


\section{Doubly-fed Induction Generator}

Doubly-fed induction generator is a special type of induction machine, with electrical brushes to control rotor currents. 

Doubly-fed induction generator coupled with three-stage gearbox (DFIG-3G) is the most common power take-off system in wind turbines, with more than half of the market share. In this configuration, the rotational speed is increased with a multi-stage gearbox. The doubly-fed induction generator rotates at speeds close to synchronous speed (1500~rpm for a 4-pole machine with 50~Hz grid).

  \begin{figure}
    \centering
    \includegraphics[width=0.5\textwidth]{DFIG_3G}
    \caption{Doubly-fed induction generator coupled to multi-stage gearbox.} 
    \label{dfig_3g}
  \end{figure}

The frequency of the armature voltage is controlled by the field winding. DFIGs differ from squirrel-cage induction generators as they need wound-rotor and electrical brushes. Armature winding of the doubly-fed induction generator is directly connected to the grid. Therefore, DFIGs do not require full-rating power electronics. The rating of the converter is about 25--30\% of the generator capacity, which reduces the cost of the system and the main advantage of the generator \cite{Li2008a} . Some of the wind turbine companies that use this configuration have been presented in Table \ref{generator_manufacturers}.

The advantages of the DFIG system can be listed as;

\begin{itemize}
	\item Partially rated power electronics.
	\item Ability to supply reactive power to the grid.
	\item Off-the-shelf components.
\end{itemize}

However, DFIGs also have a few disadvantages \cite{Li2008a}:

\begin{itemize}
	\item A multi-stage gearbox is necessary which causes heat dissipation by friction and requires regular maintenance.
	\item A slip ring is required in the generator which may cause reliability issues.
	\item High torque peaks in the machine and large stator peak currents under grid fault conditions. The power electronics should be protected.
	\item In case of grid disturbances, the ride-through capability of DFIG is required which results in complex control algorithms.
\end{itemize}

One of the largest DFIG wind turbines is manufactured by RePower, which is presented in Figure \ref{repower}. Recently, RePower made a contract with RWE Innogy to build 6~MW wind turbines, which will be installed in the Westereems wind farm in the Dutch province of Gronigen.

\begin{figure}[]
  \centering
  \includegraphics[width=0.45\textwidth]{repower_nacelle}
  \hfill
    \includegraphics[width=0.5\textwidth]{repower_farm}
\caption{RePower 6 MW wind turbines with DFIG (Courtesy of RePower, Jerome a Paris).}
  \label{repower}
\end{figure}

\section{Direct Drive}

In this topology, the generator is directly connected to the turbine (as shown in Figure \ref{eesg} and \ref{ddpmg}), so the generator speed is very low (usually in the range of 10--20~rpm). Thus, the torque requirements of direct-drive power take-off systems are very high, which results in very large and heavy generators. The large diameter of direct-drive generators means heavier support structure and large air-gap due to manufacturing tolerances.

The advantages of direct drive generators are simplified drive train and higher efficiency (especially in partial loads) \cite{Li2008a}. The absence of gearbox reduces the maintenance requirements and increases the reliability. Two types of generators are generally used with direct-drive configuration: electrically excited synchronous generators and permanent magnet generators. 

\begin{figure}
    \centering
    \includegraphics[width=0.5\textwidth]{EESG}
    \caption{Direct drive electrically excited synchronous generator.} 
    \label{eesg}
  \end{figure}

  \begin{figure}
    \centering
    \includegraphics[width=0.5\textwidth]{DDPMG}
    \caption{Direct drive permanent magnet generator.} 
    \label{ddpmg}
  \end{figure}


\subsection{Electrically-Excited Synchronous Generators}

Enercon use electrically excited synchronous 
generators in its direct-drive wind turbines, which is one of the largest generators. It is 10 m in diameter and weights 220 tones (see Figure~\ref{enercon}). 

Direct-drive electrically excited synchronous generators are similar to conventional synchronous generators with the exception that it needs to have high number of poles to compensate for the low rotational speed. The generator has a wound rotor winding which requires slip rings to excite. The reactive power and generator output voltage can be controlled using the field current. The assembly is easier compared to permanent magnet machines. The overall cost is lower than the permanent magnet generators. The disadvantages of this system can be listed as:

\begin{itemize}
	\item Largest and heaviest generator type.
	\item Slip rings require maintenance.
	\item Losses in the field winding reduce efficiency.
\end{itemize}


  \begin{figure}
    \centering
    \includegraphics[width=0.5\textwidth]{enercon}
    \caption{Enercon E-126 7.5 MW, 13 rpm direct-drive electrically excited synchronous generator (Courtesy of Enercon).} 
    \label{enercon}
  \end{figure}


\subsection{Permanent Magnet Generators}

Permanent magnet generators are a type of synchronous generator that use permanent magnets (usually rare-earth magnets) instead of field winding. The became popular in the recent years due to simple and robust structure. A direct-drive permanent magnet generator is shown in Figure \ref{switch}.

  \begin{figure}
    \centering
    \includegraphics[width=0.35\textwidth]{switch}
    \hfill
    \includegraphics[width=0.6\textwidth]{switch_turbine}
    \caption{3.8 MW, 21 rpm direct-drive permanent magnet generator and installation to the nacelle (Courtesy of The Switch and Dongfang Electrical Machinery).} 
    \label{switch}
  \end{figure}

Compared to geared solutions and direct-drive electrically excited synchronous generators; direct-drive permanent magnet generators have the following advantages:

\begin{itemize}
	\item The generator has no slip rings, thus does not require regular maintenance.
	\item No field winding losses, the generator has a better efficiency.
	\item PM generators usually have higher torque densities.
\end{itemize}

However, the drawbacks of the permanent magnet generators can be listed as;

\begin{itemize}
	\item Difficult to assembly (for iron-cored machines).
	\item High permanent-magnet prices increase the cost of the generator.
	\item Risk of demagnetization of magnets during short-circuit faults.
	\item No control over the induced voltage magnitude.
\end{itemize}

One big drawback of permanent magnet generators is the high cost of rare-earth magnets, which has increased more than tenfold between 2009 and 2011 due to export and mining regulations introduced by China  \cite{rareearthelements}. In Figure \ref{neodymium_price} the price variation of Neodymium Oxide is presented. Although, it is now much cheaper than the peak in 2011, the prices are still volatile and it is still four times more expensive compared to five years ago. It is stated in \cite{Moss2011} that the future DDPMGs are in risk due to high demand of rare-earth metals  and supply chain problems due to political risk.  

\begin{figure}[]
\centering
\includegraphics[]{neodymium_price}
\caption{Neodymium Oxide price variation between 2007 and 2013 (data points obtained from Peak Resources Ltd and Lynas Corp.).}
\label{neodymium_price}
\end{figure}

There are a few permanent magnet topologies that aim to achieve higher torque densities. One of the best candidates is the transverse flux permanent magnet (TFPM) machine. In a TFPM machine the winding space can be increased without decreasing the available space for the main flux \cite{Bang2010}. Thus, the machine can have a short pole pitch which helps to increase the frequency and magnitude of the induced voltage at low rotational speeds. In \cite{Bang2010}, Bang summarized different TFPM machines and evaluated them for direct-drive wind turbine applications. It is stated that, a 5~MW TFPM machine is  25\% lighter than a conventional PMG. In general, TFPM machines have simple windings and higher torque densities. On the other side, the flux path is more complex making the mechanical design and manufacturing more complicated \cite{Bang2009,Bang2008}.

\section{Hydraulic Power Take-off Systems}


There are two main concepts that use hydraulic components in wind turbines as presented in Figure~\ref{hydraulics}. In the first configuration, a gearbox is used to increase the rotational speed. The output speed of the hydraulic transmission is controlled so that the generator always rotates at the synchronous speed (i.e. 1500~rpm for a 4-pole machine connected to 50 Hz grid). In the second configuration, there is no mechanical gearbox and the hydraulic pump is directly coupled to the turbine. The hydraulic pump is coupled to a hydraulic motor that drives a synchronous generator at constant speed.

  \begin{figure}
    \centering
    \includegraphics[width=0.45\textwidth]{hydraulics}
    \hfill
    \includegraphics[width=0.45\textwidth]{EESG_hydraulics}
    \caption{Hydraulic system coupled with two-stage gearbox and synchronous generator and power take-off system with hydraulic transmission and synchronous generator } 
    \label{hydraulics}
  \end{figure}

Dewind uses the first configuration in a 2~MW wind turbine, which is presented in Figure \ref{voith}. A conventional synchronous generator is directly connected to the medium-voltage grid without intermediate electrical stages such as power electronics and step-up transformer. The absence of power electronics and transformer means reduced mass and increased reliability.

  \begin{figure}
    \centering
    \includegraphics[height=1.2in]{voith_windrive}
    \includegraphics[height=1.2in]{voith_schematic}
    \caption{Windrive 2~MW power take-off system with hydraulic transmission (Courtesy of Voith Turbo).} 
    \label{voith}
  \end{figure}

Artemis Intelligent Power proposed to use a hydraulic transmission system controlled by digital switches. Digital control ensures fast response and efficient operation. Artemis Intelligent Power is currently designing a 7~MW hydraulic power take-off system. The configuration has a digital displacement pump and two hydraulic motors connected to two synchronous generators. Two generator configuration introduces redundancy in the system. Thus, even if one of the generators fails the other one can continue to generate power at part-load until the next maintenance. Furthermore, at partial loads one of the sets can be idled to increase the overall efficiency.

The weak point of a hydraulic system is the fault ride through capabilities. Since there are no intermediate stages between generator and grid, in case of a grid-fault the control loop of the hydraulic system may not be fast enough to respond within grid requirements. Mechanical control systems may not be fast enough. However, Artemis Intelligent Power's electronically controlled hydraulic valves decrease the reaction time of the system. In \cite{artemis}, it is claimed that control bandwidths up to 20~Hz are achievable.

\section{Generators with a Single Stage Gearbox}

Direct-drive generators have many advantages compared to geared generators. The most obvious one is the higher availability and reduced maintenance costs. However, direct drive generators are large and heavy. The middle road between direct drive generators and high speed generators is a hybrid solution with a single-stage planetary gearbox and a medium-speed generator [31]. The generator may be a permanent magnet generator or a doubly fed induction generator. A medium-speed generator reduces the cost compared to a direct-drive generator. Moreover, the most problematic part of gearbox; the high-speed stage can be eliminated.

The idea was first introduced by Multibrid of Germany. This concept has gained much attention because it has lower generator cost than the direct-drive concept, and the lower-gear box cost, higher availability and operating reliability than the multiple-stage geared drive concept. In summary, the characteristics of the single stage gearbox concept are;

\begin{itemize}
	\item Reasonable overall reliability. The low-speed gear stage is more reliable than higher speed gear stages.
	\item Decreased generator mass and cost.
	\item Higher utilization of magnetic materials with increased rotational speed.
\end{itemize}


Thus, accepting a small risk on the gearbox side, but eliminating the largely technical risk, high mass and cost of direct drive system should result in higher overall reliability \cite{Bohmeke2003}. Areva and WinWinD are the main manufacturers of this concept (see Figure~\ref{multibrid}). Areva Multibrid 5~MW turbine has an integrated gearbox and generator design. The turbine has a single main bearing and has no main shaft. By this way, the nacelle mass is reduced and possible bearing problems are minimized. 

  \begin{figure}
    \centering
    \includegraphics[width=0.5\textwidth]{multibrid}
    \caption{Areva 5 MW Multibrid wind tutbine with combined gearbox and geratator (Courtesy of Areva).} 
    \label{multibrid}
  \end{figure}

\subsubsection{Comparison of Power Take-off Systems}


The different drive-train options have been compared in the previous section. The advantages and disadvantages of these power take-off systems have been tabulated in Table XXX.


XXX TABLO EKLE

The system cost and mass of different power take-off systems have been compared in UpWind project for turbines with a power rating of 0.75 MW, 3 MW, 10 MW [3]. The system cost includes, generator active material cost, generator construction, the gearbox (for geared PTO systems), power electronic cost and electrical subsystem cost. The generator system weight is the combined weight of generator (inactive and active materials) and gearbox. The graphs show that direct-drive generators have the highest mass and cost. The lightest solution is the permanent magnet generator connected to a multi-stage gearbox. The least expensive option is given as the doubly-fed induction generator connected to a single stage gearbox.

ELDEN GECIR

\section{Novel Power Take-off Systems}

As well as the conventional power take-off systems listed in the previous sections, there are some novel concepts that can be used for offshore wind turbines. The concepts covered in this report are:

\subsection{Ironless Permanent-Magnet Generators}

For conventional permanent magnet generators, electrical steel laminations are used in stator and rotor. The electrical steel is good to conduct magnetic flux, but it is difficult to assembly an iron cored permanent magnet generator due to high magnetic attraction forces between rotor and stator. Furthermore, the magnetic attraction forces put extra burden on bearings and mechanical structure. This attraction forces can be eliminated with an air cored stator winding design \cite{Mueller2009}. The elimination of these forces reduces the structural mass considerably \cite{McDonald2008b}. Also, the machine can be assembled and maintained much easier. 


  \begin{figure}[t]
    \centering
    \includegraphics[width=0.5\textwidth]{goliath}
    \includegraphics[width=0.45\textwidth]{25kw_cgen}
    \includegraphics[width=0.5\textwidth]{c-core_kamper}
    \caption{Air cored permanent magnet generators: 3 MW radial flux PM generator (Courtesy of Goliath Wind), 25 kW axial flux PM generator (Courtesy of NGenTec), 300 kW radial flux PM configuration \cite{Wijk2010}.} 
    \label{air_cored}
  \end{figure}


As an additional advantage, several axial flux permanent magnet machines can be stacked in the axial direction. Thus, each machine can operate independently and a faulty section can be disconnected from the rest of the machine. Thus, the machine can operate at partial load until it is maintained. 

The disadvantages of air cored permanent magnet generators can be listed as;

\begin{itemize}
	\item Increased magnet mass and cost compared to iron cored PMGs.
	\item Lower air-gap flux density.
	\item Heat transfer ratio is reduced. Additional cooling system may be required
\end{itemize}

Soft magnetic composites (SMCs) can also be used in armature coils instead of steel laminations. SMCs can be manufactured in complex shapes, giving more freedom and modularity in machine design. A spin-out company from University of Oxford, YASA (yokeless and segmented armature) motors utilized SMCs in their PM motors, which is presented in Figure \ref{yasa_motor}. The motor is optimized for high-torque electric car motors and has a forced-oil-cooling system. The motor has a high efficiency and torque density but the performance and total mass of this machine is not clear for large diameter, low-speed applications.


  \begin{figure}[t]
    \centering
    \includegraphics[width=0.4\textwidth]{yasa_motor}
    \caption{YASA (yokeless and segmented armature) motor, 750 Nm (Courtesy of YASA Motors).} 
    \label{yasa_motor}
  \end{figure}

\subsection{Pseudo Direct Drive (Magnetic Gears)}

In pseudo direct drive system, the generator is coupled to prime-mover with an intermediate magnetic transmission. The generator can be integrated with the gearbox as shown in Figure 36. Thus, the input shaft can be connected directly to slow rotating turbine, while actual speed of the generator is maintained at higher values. By this way, the size of the generator can be greatly reduced without using mechanical or hydraulic transmission.

The magnetic gears are contactless, lubricant-free and have no friction. Thus, magnetic transmission offer higher efficiency and reliability compared to geared systems. In a magnetic gear, the generator is not directly coupled with the prime mover; the vibrations are not transferred to the generator side and if input torque exceeds the maximum torque, the magnetic gears just decouple. This is advantageous for some cases, but it also limits the fault-ride-through capabilities of the system.

  \begin{figure}[t]
    \centering
    \includegraphics[width=0.4\textwidth]{magnomatics}
    \caption{Combined magnetic gearbox and generators (Courtesy of Magnomatics).} 
    \label{magnomatics}
  \end{figure}


Magnomatics is the leading company in magnetic transmission design. They are developing magnetic gears for electric car motors. Recently, Magnomatics signed a contract with UK Ministry of Defence to develop a magnetically geared propulsion system for ship propulsion. They are also planning to manufacture solutions for renewable energy. 

\subsubsection{Superconducting Generators}

Superconducting electrical machines have drawn interest since the discovery of the first superconductors in 1911, but it was after the discovery of high-temperature superconductors that the research on the applications of superconductivity has been boosted. Superconducting machines have a very high power density and can reduce the size of the machine significantly. For example in Figure \ref{amsc_36mw}, the size and mass comparison between a superconducting machine and conventional motor (ship propulsion motor) is presented.


** Add size comparison text

  \begin{figure}[t]
    \centering
    \includegraphics[width=0.8\textwidth]{amsc_ddpm_hts_comparison}
  	\caption{Size and mass comparison of direct-drive permanent magnet generator and superconducting generator for different power ratings \cite{amsc_presentation}.} 
  	\label{ddpm_hts_compare}
  \end{figure}


  \begin{figure}[t]
    \centering
    \includegraphics[width=0.45\textwidth]{36MW_AMSC}
    \includegraphics[width=0.4\textwidth]{amsc_36mw_compare}
    \caption{36.5 MW superconducting ship propulsion motor for comparison with a conventional propulsion motor US Navy and size  (Courtesy of AMSC).} 
    \label{amsc_36mw}
  \end{figure}

Some superconducting machine topologies are summarized in \cite{Kalsi2004a,Gieras2008a}. Among these machines, the most common type is a hybrid superconducting synchronous machine which is a synchronous machine with a superconducting dc-field winding. These superconducting machines have high power density but require cryogenic couplers and field excitation brushes in the rotor structure, which causes reliability problems. 

Data from direct-drive systems have been collected to compare mass to torque ratio of HTS machines with other type of generators. The result is presented in a bubble chart in Figure 38. The data for each generator can be found in \cite{Keysan2011b}. The dashed line represents ratio of generator mass to torque for permanent-magnet machines which is estimated as 25 kg/kNm by Bang et al. in \cite{Bang2008}. The Enercon-E112, which has a high mass to torque ratio (66.6 kg/kNm) is the only copper field synchronous generator in the graph. The continuous line represents the linear trend line estimated using the HTS machines in the graph.). It can be seen from the graph that HTS machines are generally lighter than PM generators for applications with torque requirements larger than 3000 kNm. 

**Figure 38 mass comparison graph

There are some companies interested in superconducting wind turbines, Converteam -now acquired by General Electric- proposed an 8 MW, 12 rpm superconducting generator, which will be 5 m in diameter and approximately 100 tonnes \cite{Lewis2007}. 

AMSC aims to manufacture a 10 MW direct-drive superconducting generator: SeaTitan which is shown in Figure 39. The generator will have an outer diameter of 4.5-5 m and have a mass of 150 tonnes \cite{Snitchler2011}.

**resmi cikar

Recently, Department of Energy (U.S.) awarded \$7.5 million to six companies to help develop next generation wind turbines \cite{dep_energy}. Two of these companies are planning to manufacture direct-drive superconducting generators. GE Global Research (in partnership with Superconductor Technologies Inc.) aims to design a 10 MW direct-drive low- temperature superconducting generator –using MgB2 wires. The generator is planned to have a stationary superconducting field winding to reduce cryogenic-coupler faults. The other company is Advanced Magnet Lab. which plans to design a 10 MW, 10 rpm, fully-superconducting generator which will be 70 tonnes. The ac losses on the armature coils can be minimized using their patented double-helix winding arrangement.

The high efficiency of superconducting machines has been emphasized in the first applications, but efficiency is not solely sufficient. In order to penetrate into the renewable energy market,  superconducting generators have to prove that they are also more reliable than alternative power take-off systems such as DDPM machines and geared induction machines \cite{Abrahamsen2010}. In a HTS machine, the cryogenic system holds the risk of decreasing the overall reliability. In particular, the cryogenic pump and couple are the most critical components. The cryogenic coupler can be eliminated by using a stationary superconducting field, which has many advantages \cite{Gieras2008a}:
\begin{itemize}
	\item No cryogenic coupler, more robust and cheap cooling system
	\item No cryogenic coupler, more robust and cheap cooling system
	\item No brushes or complex excitation systems
	\item No centrifugal forces and transient torques that can damage the superconducting material
\end{itemize}


A novel superconducting generator topology has been proposed in \cite{Keysan2011e}. A single loop-shaped stationary superconducting field winding is used in the generator. The rotating parts only consist of modular iron cores. 

**duzenle

\subsection{Generation at Sea Level}

**Hydraulics kismiyla birlestirilebilir

Lower tower top mass is favourable in larger offshore wind turbines, especially for floating platforms. Several hundred tons of nacelle mass may introduce stability issues and increases the installation cost. The tower top mass can be reduced by placing the electrical generator at sea level. This also means easier access to some critical components. The response time of the hydraulic pump and hydraulic motor should be fast enough to satisfy the fault-ride-through requirements of the grid.

ChapDrive is developing a 5 MW hydraulic power take-off system for offshore wind turbines. The project is partly funded by Statoil. Note the hydraulic motor and the generator is placed at ground level, inside of the tower. Top mass of the turbine is expected to be less than 200 tonnes.

University of Delft, applies the same hydraulic system in a slightly different in the We-at-Sea project \cite{Diepeveen2004}. Instead of using individual generators for each turbine, they propose to use a central generator(see Figure \ref{we-at-sea}) unit for all the turbines in the farm. Sea water can be used to transfer energy from turbines to central generation unit. This unit can be placed on shore for near off-shore wind turbines.

  \begin{figure}[t]
    \centering
    \includegraphics[width=0.45\textwidth]{we-at-sea}
    \caption{Schematic of the seawater-based hydraulic power take-off system  \cite{Diepeveen2004}.} 
    \label{we-at-sea}
  \end{figure}


%----------------------------------------------------------------------------------------
%	BIBLIOGRAPHY
%----------------------------------------------------------------------------------------

\bibliographystyle{unsrt}

\bibliography{../refler/offshore-wind-generators}

%----------------------------------------------------------------------------------------

\end{document}